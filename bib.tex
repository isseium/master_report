% 参考文献

\begin{thebibliography}{report}

% 第1章
\bibitem{action_program_2014} 日本政府 観光立国推進閣僚会議, 観光立国実現に向けたアクション・プログラム2014, (2014).
\bibitem{nippon_sousei} 日本創世会議 「ストップ少子化・地域元気戦略」 (2014).
\bibitem{tourism_stat} 日本旅行業協会,数字が語る旅行業2014, (2014).
\bibitem{tourism_future} 加藤弘治, 観光ビジネス未来白書, (2013).
\bibitem{kanko_hakusho_2009} 国土交通省観光庁, 平成21年度版 観光白書, (2009).
\bibitem{kanko_hakusho_2010} 国土交通省観光庁, 平成22年度版 観光白書, (2010).
\bibitem{kanko_hakusho_2011} 国土交通省観光庁, 平成23年度版 観光白書, (2011).
\bibitem{kanko_hakusho_2012} 国土交通省観光庁, 平成24年度版 観光白書, (2012).
\bibitem{kanko_hakusho_2013} 国土交通省観光庁, 平成25年度版 観光白書, (2013).
\bibitem{kanko_hakusho_2014} 国土交通省観光庁, 平成26年度版 観光白書, (2014).
\bibitem{wakamono_shinko} 国土交通省観光庁, 若者旅行振興の必要性について, (2011).
\bibitem{define_of_recommendation_system} J. A. Konstan and J. Riedl. Recommender systems: Collaborating in commerce and communities. In Tutorial at ACM CHI2003, (2003).
\bibitem{kamishima_recommendation} 神嶌敏弘, 「推薦システムのアルゴリズム」, (2014).
\bibitem{information_overload} P. Maes. Agents that reduce work and information overload. Communications of ACM, Vol. 37, No. 7, pp. 30-40, (1994).

% 第2章
\bibitem{japanese_society_list} Hirokazu Kobayashi, "観光学術学会が発足!(2012/04/30投稿)", ツーリズムマーケティングのフィールドから, http://hirokazukobayashi.air-nifty.com/marketing/2012/04/post-8570.html, (2014/12/14 参照)
\bibitem{society_tourism} 日本観光学会, http://www.kankoga.or.jp/
\bibitem{society_hm} 日本ホスピタリティ・マネジメント学会, http://www.hospitality.gr.jp/
\bibitem{society_jafit} 日本国際観光学会, http://www.jafit.jp/
\bibitem{society_yoka} 余暇ツーリズム学会, http://www.yokagakkai.jp/
\bibitem{society_afz} 総合観光学会, http://www.afz.jp/~skankou/
\bibitem{society_m} 観光まちづくり学会, http://www.kankou-m.jp/
\bibitem{society_jsthe} 日本観光ホスピタリティ教育学会, http://jsthe.org/
\bibitem{society_sti} 観光情報学会, http://www.sti-jpn.org/
\bibitem{society_niu} 長崎国際大学 国際観光学会, http://www1.niu.ac.jp/sits/
\bibitem{society_jatf} 日豪ツーリズム学会, http://www.jatf.net/
\bibitem{society_iatm} 国際観光医療学会, http://www.iatm.jp/
\bibitem{society_duan} 日中国際ツーリズム学会, http://npo.duan.jp/qd.htm
\bibitem{society_ct} コンテンツツーリズム学会, http://contentstourism.com/
\bibitem{society_jsts} 観光学術学会, http://jsts.sc/
\bibitem{kanko_define} 今井 TODO
\bibitem{yokomizo_1998} 溝尾良隆, 「観光・観光資源・観光地の定義 」, 観光研究, Vol.9, No.2, p.36, (1998).
\bibitem{toshin_1995} 建設省観光政策審議会, 今後の観光政策の基本的な方向について(答申39号), (1995).
\bibitem{tourism_law_2007} 日本国, 観光立国推進基本法, (2007).
\bibitem{tourism_kensetsu_1974} 建設省道路局, 観光レクリエーション交通調査, (1974)
\bibitem{recommendation_type_of_personalize} J. Ben Schafer, J. A. Konstan, and J. Riedl. E-commerce recommendation applications. Data Mining and Knowledge Discovery, Vol. 5, pp. 115-153, (2001).
\bibitem{tarui} 樽井 勇之, 協調フィルタリングとコンテンツ分析を利用した観光地推薦手法の検討,上武大学紀要集2001, 第36号, pp.1-14, (2011).
\bibitem{ctplanner} 倉田陽平, 対話型観光プランニングシステムに向けて, 第18回地理情報システム学会学術大会, 地理情報システム学会講演論文集 18, (2009).
\bibitem{ctplanner2} 倉田陽平, 有馬貴之, 対話的旅行計画作成支援システムの実装と評価, 第25回日本観光研究学会全国大会, 日本観光研究学会全国大会学術論文集 25, pp.173-176, (2010).
\bibitem{ctplanner3} 倉田陽平, Web上での対話的な旅行プラン作成支援. 情報処理学会第74回全国大会, (2012年).
\bibitem{ctplanner3b} 倉田陽平, CT-Planer 3: Web上での対話的な旅行プラン作成支援, 観光科学研究 5, pp.159-165, (2012). 
\bibitem{ctplanner4} 旅行プラン作成支援ツールCT-Planner4の留学生を対象としたモニター調査, 観光情報学会第10回全国大会, pp.56-57, (2013).
\bibitem{ctplanner5} 倉田陽平, 原辰徳, インターネット上での対話的旅行プラン作成支援サービスとその展開可能性, サービス学会第2回国内大会, pp.191-194, (2014).
\bibitem{ctplanner_web} CT-Planner, http://ctplanner.jp/ctp5/

% 第4章
\bibitem{yolp_rgc} ヤフー株式会社, Yahoo!リバースジオコーダー, \\ http://developer.yahoo.co.jp/webapi/map/openlocalplatform/v1/reversegeocoder.html, (2014/12/19 参照)
\end{thebibliography}
