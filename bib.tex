% 参考文献

\begin{thebibliography}{report}
\bibitem{action_program_2014} 日本政府 観光立国推進閣僚会議, 観光立国実現に向けたアクション・プログラム2014, (2014).
\bibitem{nippon_sousei} 日本創世会議 「ストップ少子化・地域元気戦略」 (2014).
\bibitem{tourism_stat} 日本旅行業協会,数字が語る旅行業2014, (2014).
\bibitem{tourism_future} 加藤弘治, 観光ビジネス未来白書, (2013).
\bibitem{kanko_hakusho_2009} 国土交通省観光庁, 平成21年度版 観光白書, (2009).
\bibitem{kanko_hakusho_2010} 国土交通省観光庁, 平成22年度版 観光白書, (2010).
\bibitem{kanko_hakusho_2011} 国土交通省観光庁, 平成23年度版 観光白書, (2011).
\bibitem{kanko_hakusho_2012} 国土交通省観光庁, 平成24年度版 観光白書, (2012).
\bibitem{kanko_hakusho_2013} 国土交通省観光庁, 平成25年度版 観光白書, (2013).
\bibitem{kanko_hakusho_2014} 国土交通省観光庁, 平成26年度版 観光白書, (2014).
\bibitem{wakamono_shinko} 国土交通省観光庁, 若者旅行振興の必要性について, (2011).
\bibitem{define_of_recommendation_system} J. A. Konstan and J. Riedl. Recommender systems: Collaborating in commerce and communities. In Tutorial at ACM CHI2003, (2003).
\bibitem{kamishima_recommendation} 神嶌敏弘, 「推薦システムのアルゴリズム」, (2014).
\bibitem{information_overload} P. Maes. Agents that reduce work and information overload. Communications of ACM, Vol. 37, No. 7, pp. 30-40, (1994).
\bibitem{kanko_define} 今井
\bibitem{japanese_society_list} Hirokazu Kobayashi, "観光学術学会が発足!(2012/04/30投稿)", ツーリズムマーケティングのフィールドから, http://hirokazukobayashi.air-nifty.com/marketing/2012/04/post-8570.html, (2014/12/14 参照)
\bibitem{yokomizo_1998} 溝尾良隆, 「観光・観光資源・観光地の定義 」, 観光研究, Vol.9, No.2, p.36, (1998).
\bibitem{toshin_1995} 建設省観光政策審議会, 今後の観光政策の基本的な方向について(答申39号), (1995).
\bibitem{tourism_law_2007} 日本国, 観光立国推進基本法, (2007).
\bibitem{tourism_kensetsu_1974} 建設省道路局, 観光レクリエーション交通調査, (1974)
\bibitem{recommendation_type_of_personalize} J. Ben Schafer, J. A. Konstan, and J. Riedl. E-commerce recommendation applications. Data Mining and Knowledge Discovery, Vol. 5, pp. 115-153, (2001).
\end{thebibliography}
